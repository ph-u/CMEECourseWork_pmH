%% Author: ph-u
%% Script: autocorrelation.Rnw
%% Desc: pdf report creation for an annual temperature `Rdata` dataset
%% Input: None -- need to "Compile PDF" within RStudio.app
%% Output: pdf report in `Code` subdirectory (and other auto-generated files)
%% Arguments: 0
%% Date: Oct 2019

\documentclass{article}

\usepackage[margin=1in]{geometry}
\usepackage{amsmath,hyperref}

\title{Question: Autocorrelation in weather}
\author{ph-u (CID: 01786076)}
\date{}

\usepackage{Sweave}
\begin{document}
\input{autocorrelation-concordance}

\maketitle
\begin{center}
  Hypothesis: Annual temperature ($^o$C) is influenced by the previous year.
\end{center}
\section{Load \textbf{KeyWestAnnualMeanTemperature.Rdata}}
\begin{Schunk}
\begin{Sinput}
> load("../Data/KeyWestAnnualMeanTemperature.RData");ls()
\end{Sinput}
\begin{Soutput}
[1] "ats"
\end{Soutput}
\end{Schunk}

\section{Examine correlation coefficient of data\label{osp}}
\begin{Schunk}
\begin{Sinput}
> print(b<-unlist(cor(ats,method = "spearman"))[1,2])
\end{Sinput}
\begin{Soutput}
[1] 0.5255559
\end{Soutput}
\end{Schunk}

\section{Plot data\label{plot}}
\includegraphics{autocorrelation-004}
\\
Pairing annual temperature by shifting yearly data by 1\\
\begin{Schunk}
\begin{Sinput}
> ats.0<-data.frame(ats[,2][1:dim(ats)[1]-1],ats[,2][2:dim(ats)[1]])
\end{Sinput}
\end{Schunk}
\clearpage
And plot the paired data\\
\includegraphics{autocorrelation-006}

\section{Sample Spearman correlation 10K times by random shuffles\label{ssp}}
\begin{Schunk}
\begin{Sinput}
> dm<-1e4
> a<-rep(NA,dm)
> i<-1
> for(x in 1:dm){
+   
+   ## shuffle data into random pairs
+   ats.0<-sample(ats[,2],dim(ats)[1],replace = F)
+   ats.0<-data.frame(ats.0[1:length(ats.0)-1],ats.0[2:length(ats.0)])
+   
+   ## Spearman correlation on newly-shuffled pairs
+   a[i]<-unlist(cor(ats.0,method = "spearman"))[1,2]
+   i<-i+1
+ }
\end{Sinput}
\end{Schunk}
With Spearman correlation coefficient mean (from sampling) calculated as:\\
\begin{Schunk}
\begin{Sinput}
> mean(a)
\end{Sinput}
\begin{Soutput}
[1] -0.009002796
\end{Soutput}
\end{Schunk}
\section{Fraction of sampling $>$ overall coefficient (approx. p.val)\label{pval}}
\includegraphics{autocorrelation-010}
\begin{Schunk}
\begin{Sinput}
> length(a[which(a>b)])/length(a)
\end{Sinput}
\begin{Soutput}
[1] 0
\end{Soutput}
\end{Schunk}
\section{Discussion}
Correlation coefficient from overall (Sec.\ref{osp}) and sampled (Sec.\ref{ssp}) were different with strong statistical significance (Sec.\ref{pval}, p$<<$0.01).  It showed that the current year temperature is influenced/correlated with previous year's.
%Correlation coefficient from both overall (Sec.\ref{osp}) and sampled (Sec.\ref{ssp}) were only in medium levels.  The approximated p-value (Sec.\ref{pval}) is falsifying the hypothesis (p$>>$0.05).\\\\
%The results showed that time (i.e. year) is not a statistically-significant factor for the annual temperature for Florida in the twentieth century.  Hence the ``best-fitted" curve in the plot (Sec.\ref{plot}) was a mis-interpretation.  Other factors (including but not limited to atmospheric carbon dioxide levels, atmospheric sulphur dioxide levels and suspended particulates level) should also be considered in future analyses.

\end{document}
