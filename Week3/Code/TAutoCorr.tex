%% Author: PokMan Ho pok.ho19@imperial.ac.uk
%% Script: TAutoCorr.Rnw
%% Desc: pdf report creation for a annual temperature `Rdata` dataset
%% Input: None -- need to `Compile PDF` within RStudio.app
%% Output: pdf report in `Code` subdirectory (and other auto-generated files)
%% Arguments: 0
%% Date: Oct 2019

\documentclass{article}

\usepackage[margin=1in]{geometry}
\usepackage{amsmath,hyperref}

\title{Question: Autocorrelation in weather}
\author{PokMan Ho (CID: 01786076)}
\date{}

\usepackage{Sweave}
\begin{document}
\input{TAutoCorr-concordance}

\maketitle
\begin{center}
  Hypothesis: Annual temperature ($^o$C) is influenced by the previous year.
\end{center}
\section{Load \textbf{KeyWestAnnualMeanTemperature.Rdata}}
\begin{Schunk}
\begin{Sinput}
> load("../Data/KeyWestAnnualMeanTemperature.RData");ls()
\end{Sinput}
\begin{Soutput}
[1] "ats"
\end{Soutput}
\end{Schunk}

\section{Examine correlation coefficient of data\label{osp}}
\begin{Schunk}
\begin{Sinput}
> print(b<-unlist(cor(ats,method = "spearman"))[1,2])
\end{Sinput}
\begin{Soutput}
[1] 0.5255559
\end{Soutput}
\end{Schunk}

\section{Plot data\label{plot}}
\includegraphics{TAutoCorr-004}

\section{Sample Spearman correlation 10K times through random time-series perturbation\label{ssp}}
\begin{Schunk}
\begin{Sinput}
> dm<-1e4
> a<-rep(NA,dm);i<-1
> for(x in sample((2:dim(ats)[1]),dm,replace = T)){
+   ## random pick 10K sample from years range (1901-2000)
+   a[i]<-unlist(cor(ats[(1:x),],method = "spearman"))[1,2]
+   i<-i+1}
\end{Sinput}
\end{Schunk}
\clearpage
With Spearman correlation coefficient mean (from sampling) calculated as:
\begin{Schunk}
\begin{Sinput}
> mean(a)
\end{Sinput}
\begin{Soutput}
[1] 0.5355228
\end{Soutput}
\end{Schunk}
\section{Fraction of sampling $>$ overall coefficient (approx. p.val)\label{pval}}
\begin{Schunk}
\begin{Sinput}
> length(a[which(a>b)])/length(a)
\end{Sinput}
\begin{Soutput}
[1] 0.4887
\end{Soutput}
\end{Schunk}
\section{Discussion}
Correlation coefficient from both overall (Sec.\ref{osp}) and sampled (Sec.\ref{ssp}) were only in medium levels.  The approximated p-value (Sec.\ref{pval}) is falsifying the hypothesis (p$>>$0.05).\\\\
The results showed that time (i.e. year) is not a statistically-significant factor for the annual temperature for Florida in the twentieth century.  Hence the ``best-fitted" curve in the plot (Sec.\ref{plot}) was a mis-interpretation.  Other factors (including but not limited to atmospheric carbon dioxide levels, atmospheric sulphur dioxide levels and suspended particulates level) should also be considered in future analyses.

\end{document}
